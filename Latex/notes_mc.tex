\documentclass{article}
\usepackage{amsmath,amssymb}
\begin{document}

    \title{Monte-Carlo Methods for Parabolic PDEs -- Core Conclusions}

    \subsection*{1. PDE-SDE Correspondence}

    \begin{itemize}
        \item Linear second-order parabolic PDEs correspond to SDEs through the shared generator
        \[
            \mathcal L = a(x)\,\partial_x + \tfrac12 \sigma^2(x)\,\partial_{xx}.
        \]
        \item \textbf{Forward direction (SDE $\rightarrow$ Fokker--Planck):}
        The density of the SDE satisfies the Fokker--Planck equation
        \[
            \partial_t p = \mathcal L^\ast p.
        \]
        \item \textbf{Backward direction (PDE $\rightarrow$ SDE):}
        Feynman--Kac represents the PDE solution as an expectation over the associated SDE.
    \end{itemize}

    \subsection*{2. Monte-Carlo Solution Strategy}

    \begin{enumerate}
        \item Map the PDE to its associated SDE.
        \item Simulate many independent SDE trajectories.
        \item Compute the functional dictated by the PDE (terminal payoff, exponential weight, or empirical density).
        \item Average over samples.
    \end{enumerate}

    \subsection*{3. Distinguish Two PDE Types}

    \paragraph{Fokker--Planck PDE.}
    The solution is a probability density.  
    Solve via Monte-Carlo by simulating the SDE and estimating the density of endpoints using a histogram or kernel estimator.

    \paragraph{General linear parabolic PDE.}
    The solution is not a density.  
    Monte-Carlo uses the Feynman--Kac representation:
    \[
    u(x,t) = \mathbb E\Big[f(X_T)\exp\Big(-\int_t^T V(X_s)\,ds\Big)\,\Big|\,X_t = x\Big].
    \]

    \subsection*{4. Euler--Maruyama vs. Milstein}

    \begin{itemize}
        \item \textbf{Euler--Maruyama:} strong order $1/2$, weak order $1$; simplest and cheapest.
        \item \textbf{Milstein:} strong order $1$ (higher), weak order $1$ (same as EM); requires $\sigma'(x)$.
    \end{itemize}

    For Fokker--Planck Monte-Carlo:
    \begin{itemize}
        \item The goal is a \emph{weak} quantity (density/expectation).
        \item Both schemes have weak order $1$, so Milstein does not improve the convergence rate of the density.
        \item \textbf{Default choice: Euler--Maruyama.}
    \end{itemize}

    \subsection*{5. Practical Notes}

    \begin{itemize}
        \item Boundary handling (absorbing/reflecting) is often the dominant source of bias.
        \item If positivity must be preserved, naive EM/Milstein may fail; use tailored schemes.
        \item For density estimation, investing in more samples or smaller step size is typically more effective than using Milstein.
    \end{itemize}

\end{document}